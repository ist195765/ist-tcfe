\newpage
\section{Introduction}
\label{sec:introduction}
% state the learning objective 

In this laboratory assignment we study a circuit (Fig. 1) containing various elements, to be more specific, 1 capacitor, 7 resistances, 1 sinusoidal voltage source $V_s$, 1 dependent source of voltage and 1 dependent source of current. In order to study this circuit we use various methods, such as the node method, we change the circuit so we can find various variables associated with it, like the $R_{eq}$ and the total solution of $V_6$ and also study the frequency response on the main circuit.\\

The sinusoidal voltage source $V_s$ varies in time exactly as it follows:

\begin{equation}
V_{s}(t) = V_{s}u(-t) + sin(2 \pi f t)u(t)
\label{equation1}
\end{equation}
Where
\begin{equation}
 u(t)= e
    \begin{cases}
     0 & \text{t $<$ 0}\\
      1 & \text{t $\geq$ 0}
    \end{cases} 
\label{eq:tt}      
\end{equation}


The next table displays the data generated automatically by the Python Script:

\begin{table}[H] \centering
\begin{tabular}{|
>{\columncolor[HTML]{FFCC67}}l |c|}
\hline
\multicolumn{2}{|l|}{\cellcolor[HTML]{EABD8B}Octave - Voltages (V)} \\ \hline
R1 & 1.013609e+03 Ohm\\ \hline
R2 & 2.016578e+03 Ohm\\ \hline
R3 & 3.006816e+03 Ohm\\ \hline
R4 & 4.049229e+03 Ohm\\ \hline
R5 & 3.053925e+03 Ohm\\ \hline
R6 & 2.092502e+03 Ohm\\ \hline
R7 & 1.022320e+03 Ohm\\ \hline
C & 1.029587e-06 F\\ \hline
Kb & 7.213324e-03 A/V \\ \hline
Kd & 8.321035e+03 V/A \\ \hline

\end{tabular}
\caption{Initial data}
\end{table}


\begin{figure}[H] 

\centering
\includegraphics[width=\textwidth]{circuit1.pdf}
\caption{The RC Circuit}
\label{fig:first}

\end{figure}


In Section~\ref{sec:analysis}, a theoretical analysis of the circuit is
presented. In Section~\ref{sec:simulation}, the circuit is analysed by
simulation, and the results are compared to the theoretical results obtained in
Section~\ref{sec:analysis}. The conclusions of this study are outlined in
Section~\ref{sec:conclusion}. \\


