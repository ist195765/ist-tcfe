\section{Simulation Analysis}
\label{sec:simulation}

\subsection{Operating Point Analysis}

Table~\ref{tab:op} shows the simulated operating point results for the circuit
under analysis. Compared to the theoretical analysis results, one notices the
following differences: describe and explain the differences.

\begin{table}[h]
  \centering
  \begin{tabular}{|l|r|}
    \hline    
    {\bf Name} & {\bf Value [A or V]} \\ \hline
    @cb[i] & 0.000000e+00\\ \hline
@ce[i] & 0.000000e+00\\ \hline
@q1[ib] & 7.022567e-05\\ \hline
@q1[ic] & 1.404513e-02\\ \hline
@q1[ie] & -1.41154e-02\\ \hline
@q1[is] & 5.765392e-12\\ \hline
@rc[i] & 1.411536e-02\\ \hline
@re[i] & 1.411536e-02\\ \hline
@rf[i] & 7.022567e-05\\ \hline
@rs[i] & 0.000000e+00\\ \hline
v(1) & 0.000000e+00\\ \hline
v(2) & 0.000000e+00\\ \hline
base & 2.254108e+00\\ \hline
coll & 5.765392e+00\\ \hline
emit & 1.411536e+00\\ \hline
vcc & 1.000000e+01\\ \hline

  \end{tabular}
  \caption{Operating point. A variable preceded by @ is of type {\em current}
    and expressed in Ampere; other variables are of type {\it voltage} and expressed in
    Volt.}
  \label{tab:op}
\end{table}



\subsection{Transient Analysis}

Figure~ shows the simulated transient analysis results for the
circuit under analysis. Compared to the theoretical analysis results, one
notices the following differences: describe and explain the differences.





\subsection{Frequency Analysis}

\subsubsection{Magnitude Response}

Figure~ shows the magnitude of the frequency response for the
circuit under analysis. Compared to the theoretical analysis results, one
notices the following differences: describe and explain the differences.


\subsubsection{Phase Response}

Figure~ shows the magnitude of the frequency response for the
circuit under analysis. Compared to the theoretical analysis results, one
notices the following differences: describe and explain the differences.


\subsubsection{Input Impedance}

Figure~ shows the magnitude of the frequency response for the
circuit under analysis. Compared to the theoretical analysis results, one
notices the following differences: describe and explain the differences.




